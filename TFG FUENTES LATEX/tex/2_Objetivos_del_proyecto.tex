\capitulo{2}{Objetivos del proyecto}

El objetivo principal buscado en este proyecto es el de realizar una aplicación software que permita al usuario el establecer comunicación entre él mismo y un dispositivo robótico, por medio de internet y usando como nexo entre ambos la realidad virtual y todo lo que nos ofrece.

En conjunto con ITCL\cite{itcl} pusimos sobre la mesa una serie de objetivos que podemos considerar como principales, ya que son aquellos que debíamos cumplir para sacar adelante el proyecto, y otros más ambiciosos con mayor dificultad y requerimiento a nivel de tiempo que podemos considerar secundarios. Por otro lado marcamos unos también a nivel técnico y yo por mi parte, unos objetivos personales a cumplir con el proyecto.

\section{Objetivos Principales}
\begin{enumerate}
    \item Comprender y ver el funcionamiento de la plataforma de desarrollo Unity\cite{Unity}.
    \item Comprender las posibilidades que nos ofrecen las gafas de realidad virtual Oculus Quest 2\cite{Quest2}.
    \item Entender el funcionamiento de los guantes hápticos SenseGlove Nova.
    \item Establecer la conexión entre los guantes y las gafas, con nuestro ordenador, para la recolección de datos con Unity.
    \item Desarrollo de software especializado con Unity que de sentido a esos datos recogidos.
    \item Conexión de Unity y nuestro ordenador con el robot Kinova a través de la red pública.
    \item Desarrollo de software para teleoperar todas las partes móviles del robot mediante nuestra tecnología de realidad virtual. 
\end{enumerate}

\section{Objetivos Secundarios}
\begin{enumerate}
    \item Conseguir, además del desplazamiento en el espacio de la pinza robot, implementar la funcionalidad de apertura y cierre de la pinza con gestos desde los guantes.
    \item Recibir respuesta háptica\cite{Haptica1} al operador en relación a variaciones en el estado del robot o su entorno.
    \item Visualizar de forma estereoscópica lo que está viendo el robot.
    \item Comandar la cámara estereoscópica del robot únicamente con los movimientos de cabeza del operador.
\end{enumerate}

\section{Objetivos Técnicos}
\begin{enumerate}
    \item Establecer la conexión inalámbrica entre todos los dispositivos involucrados en el proyecto.
    \item Aprender y entender como comunicarse con el robot.
    \item Uso de repositorios y herramientas tanto de control de versiones como de planificación/administración del proyecto.
    \item Establecer un material bien documentado, que pueda ser de utilidad a las personas interesadas en esta clase de proyectos.
\end{enumerate}

\newpage

\section{Objetivos Personales}
\begin{enumerate}
    \item Aprender a desarrollar con Unity\cite{Unity} aplicaciones para Oculus Quest 2.
    \item Aprender a desarrollar y entender el mundo de la realidad virtual\cite{VR}.
    \item Desarrollar software de manera depurada y limpia.
    \item Aprovechar los conocimientos recibidos durante la carrera en diferentes ámbitos, para fusionarlos en este proyecto.
    \item Uso de la metodología agil de desarrollo software SCRUM\cite{SCRUM}
    \item Desarrollar la documentación del proyecto mediante \LaTeX.
\end{enumerate}