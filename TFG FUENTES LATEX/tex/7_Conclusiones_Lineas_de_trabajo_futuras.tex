\capitulo{7}{Conclusiones y Líneas de trabajo futuras}
Para finalizar tanto esta memoria como el proyecto destinado a ser mi trabajo de fin de grado de la carrera de Ingeniería Informática vamos a llevar a cabo una pequeña conclusión y también veremos algunas de las posibles líneas de trabajo y de investigación futuras.

\section{Conclusiones}
Llegados a este punto podemos hablar de una conclusión de los objetivos de manera correcta. Satisfaciendo tanto lo esperado como lo marcado por los requisitos para el software del proyecto. Por la naturaleza del proyecto, el cual es eminentemente de investigación podríamos haber topado con problemas y errores que no nos permitieran alcanzar todo lo marcado. En cambio podemos afirmar que se encuentra desarrollado un software a través del motor\cite{MotorGrafico} gráfico Unity\cite{Unity}, que nos permite conectar un sistema de dispositivos de realidad virtual\cite{VR} compuesto por unas gafas, Oculus Quest 2 con sus controladores, y unos guantes hápticos Nova con un robot para ser teleoperado mediante nuestra mano y a través de gestos.

Se alcanzó a conseguir todos los objetivos marcados como principales, algunos de los secundarios, los marcados de carácter técnico y de los que personalmente me siento muy orgulloso que son los objetivos marcados como personales.

Todo este proceso de desarrollo del proyecto ha sido concluido con éxito, pero por el camino he enfrentado momentos más complicados, principalmente al principio, ya que no tenía la experiencia suficiente de trabajo con las herramientas con las que ha sido desarrollado el proyecto ni con las metodologías aplicadas.
Considero con un carácter fundamental el hecho de haber cursado la mayor parte de un curso de desarrollo con Unity antes de empezar el proyecto, ya que sin esta formación me hubiera sido imposible desarrollar.

Las 300 horas marcadas para la resolución del mismo me han resultado justas, ya que en caso de no haber podido completar a tiempo alguno de los sprints, podría haberme visto sin tiempo para concluir el desarrollo de la manera esperada. Además estas 300 horas han sido en un 95\% de su totalidad dedicadas al desarrollo, el resto a documentar y elaborar tanto esta memoria como el anexo. Por lo tanto me he visto necesitando de muchas horas más por mi cuenta para poder acabar estos documentos.

No tenía ninguna experiencia previa trabajando ni con Unity, ni con C\#, ni con robots, ni con tecnologías de realidad virtual. Pero puedo asegurar que esta experiencia de trabajo ha sido positiva, enriquecedora y satisfactoria para mi. También destaco el uso de \LaTeX para documentar, ya que es una herramienta que permite generar documentación de una manera sencilla pero estéticamente perfecta.

Desde este punto debo dar las gracias tanto a la Universidad de Burgos como a ITCL, ya que sin ambos dos no hubiera podido adentrarme en un proyecto de esta magnitud, con un elevado coste presupuestario y un desarrollo tan especial. Dar las gracias también a mi tutor empresarial Alejandro Langarica Aparicio ya que ha sido un gran mentor y ejemplo a la hora de trabajar y resolver problemas, y a Carlos Cambra Baseca por ser mi tutor universitario de este TFG.

Además quisiera aprovechar este espacio para dar las gracias al resto de profesores que me han formado para poder alcanzar este momento y esta entrega. 
Indudablemente debo agradecer a mis padres, a mi hermano Pablo, a Julia, a los vatos, a mis amigos cercanos y a todo aquel que me dio fuerza para llevar a cabo este trabajo, por ser una fuente tanto de inspiración como de energía para mi.

\newpage

\section{Líneas de Trabajo Futuras}
Algunas de las posibilidades de trabajo de cara al futuro que se contemplan desde este punto del proyecto son:

\begin{itemize}
    \item Recibir respuesta háptica al operador en relación a variaciones en
el estado del robot o su entorno.
    \item Desarrollar un sistema de visión estereoscópico ajustable y sincronizado al movimiento de cabeza del operador
    \item Desarrollar una interfaz para superponer sobre la imagen virtual que tiene el teleoperador, en vista de dotar de más funcionalidades.
    \item Proyectar un sistema inalámbrico de comunicación basado en la nube para el control remoto.
\end{itemize}