\capitulo{1}{Introducción}
En pleno siglo XXI, hemos podido ir observando como casi día a día las nuevas soluciones tecnológicas desarrolladas van cubriendo más y más necesidades de nuestra sociedad. Estas soluciones también avanzan en cuanto a la manera de ser controladas y gestionadas. Hasta la fecha la gran mayoría de los robots que son utilizados de manera teleoperada han sido controlados a través de joysticks, paneles de mandos o teclados. En cuanto a la visión, la manera de gestionarla convencionalmente ha sido desde monitores o pantallas. 

A día de hoy, la tecnología nos da la posibilidad de vivir esta experiencia de control a distancia de una forma mucho más inmersiva y es ahí donde está el motivo que da alas a este proyecto. Se realizó un estudio del interés posible por las empresas en este tipo de tecnologías y este fue muy amplio. El motivo de este interés proviene del hecho de que hay tareas que un robot con inteligencia artificial no es capaz de hacer de forma autónoma por la complejidad que requiere y por otro lado, son peligrosas de realizar por el ser humano. Ejemplos son tareas como la de colocar cargas explosivas en una mina, desactivar bombas, inspeccionar o tomar muestras en ambientes tóxicos, reparaciones en entornos subacuáticos o incluso, el mantenimiento de satélites en órbita. En todas estas tareas se usan robots teleoperados y cuanto mayor sea la inmersividad para el operario, mejor será desempeñada la tarea.

La idea propuesta por este TFG y desarrollado en conjunto con el Instituto Tecnológico de Castilla y León (ITCL) \cite{ITCL} nace de una idea genuina de mi tutor en la empresa, Alejandro Langarica. Y busca con él, explorar las posibilidades que nos da la tecnología para el control robótico haciendo uso de tecnología inmersiva, con dispositivos de realidad virtual\cite{VR} como Oculus Quest 2 \cite{Quest2} y los guantes hápticos Nova que usaremos en el proyecto.

\section{Estructura de la memoria}
A la hora de realizar la memoria del proyecto, este se ha dividido en los siguientes apartados para una mejor y más fácil comprensión de todo lo expuesto:
\begin{itemize}
\item \textbf{Introducción} : Ligera descripción del proyecto junto a la estructura tanto de la memoria como de los anexos.
\item \textbf{Objetivos del proyecto}: Explicación breve de los objetivos perseguidos en el desarrollo del proyecto, distinguiéndolos en principales, secundarios, técnicos y personales. 
\item \textbf{Conceptos teóricos}: Explicaciones necesarias para comprender aquellos conceptos dispuestos y utilizados en la memoria.
\item \textbf{Técnicas y herramientas}: Presentación de las técnicas, metodologías y herramientas empleadas para realizar el proyecto.
\item \textbf{Aspectos relevantes del desarrollo}: Presentación de aquellas partes más importantes y destacadas a la hora de llevar a cabo el desarrollo del proyecto.
\item \textbf{Trabajos relacionados}: Apartado donde se explican aquellos proyectos realizados y similares al expuesto en esta memoria.
\item \textbf{Conclusiones y lineas de trabajo futuras}: Explicación de aquellas conclusiones derivadas del desarrollo del proyecto, junto con la descripción de posibilidades de mejora y expansión del mismo en el futuro.
\end{itemize}


\section{Anexos}
\begin{itemize}
    \item \textbf{Plan de proyecto software} : Planificación temporal junto con los estudios de viabilidad.
    \item \textbf{Especificación de requisitos}:  Catálogo y especificación de los requisitos necesarios para desarrollar el proyecto. 
    \item \textbf{Diseño}: Aquello relacionado con el diseño del software.
    \item \textbf{Manual del programador}: Guía para la utilización de los ficheros para trabajar con ellos o continuar desarrollando.
    \item \textbf{Manual del usuario}: Guía para el uso del software a nivel de usuario.
\end{itemize}
\subsection{Materiales Añadidos}
La entrega del material correspondiente a este trabajo final de grado está compuesta por la memoria, los anexos, la carpeta contenedora del proyecto, fuentes de la documentación a través de un directorio con el proyecto de \LaTeX y un vídeo que recoge una muestra de algunos de los resultados.