\capitulo{6}{Trabajos relacionados}

La teleoperación nació junto con la industria nuclear en los años 50 debido a la necesidad de manejar materiales altamente radioactivos , muy peligrosos para la salud simplemente con estar en su presencia. 
En los años setenta la teleoperación alcanzó su madurez con su aplicación en misiones espaciales, concretamente en los vehículos a control remoto Lunojod I y Lunojod II enviados a la Luna en 1970 y 1973 respectivamente.

En las últimas décadas, los avances en teleoperación han ido muy ligados a la evolución de la robótica y la informática. Gracias a estos avances, muchos sistemas de teleoperación actuales consisten en robots que tienen una gran autonomía y solo precisan ser teleoperados para determinadas acciones que, debido a las limitaciones de la robótica, no pueden realizar por si solos. También se ha progresado en las interfaces hombre-máquina buscando una mayor sensación de control de la máquina y de telepresencia.

Los robots tele-operados pueden encontrarse en la industria nuclear (mantenimiento de reactores), química (manejo a distancia de sustancias peligrosas o tóxicas), militar (detección, manipulación y desmantelamiento de cargas explosivas), desminado humanitario, espacial (exploraciones realizadas en la luna y en marte, también en transbordadores espaciales), minera (excavaciones, manejo de cargas explosivas en minas y túneles), en el sector de seguridad, mantenimiento y rescate (inspección de sistemas de alcantarillado y tuberías, reconocimiento de zonas de desastres), telecirugía , entre muchas otras áreas.

\newpage
Algunos ejemplos de proyectos realizados son:
\begin{itemize}
    \item \textbf{Proyecto H2020 Sarafun}\cite{Sarafun} : se investigaron las habilidades sensoriales y cognitivas de vanguardia para un robot, así como habilidades de razonamiento necesarias para planificar y ejecutar una tarea de ensamblaje.
    \item \textbf{Proyecto JAK\&Teleoperación}\cite{JAK} : se planteó un sistema que permite al usuario describir y controlar el robot desde una simulación basada en realidad virtual
    \item La compañía \textbf{SET}\cite{Set}  está desarrollando un sistema para operar robots remotamente que permitiría operar estos elementos aunque estén en el espacio
\end{itemize}
